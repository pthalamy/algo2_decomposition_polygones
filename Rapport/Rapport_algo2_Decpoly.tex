\documentclass[10.9pt]{article}

\usepackage[french]{babel}
\usepackage[utf8]{inputenc}
\usepackage{fancyhdr}
\usepackage{lastpage}
\usepackage{graphicx}
\usepackage[T1]{fontenc}
\usepackage{amsmath,amssymb}
\usepackage{fullpage}
\usepackage{url}
\usepackage{xspace}
 
\usepackage{listings}
\lstset{
  morekeywords={abort,abs,accept,access,all,and,array,at,begin,body,
      case,constant,declare,delay,delta,digits,do,else,elsif,end,entry,
      exception,exit,for,function,generic,goto,if,in,is,limited,loop,
      mod,new,not,null,of,or,others,out,package,pragma,private,
      procedure,raise,range,record,rem,renames,return,reverse,select,
      separate,subtype,task,terminate,then,type,use,when,while,with,
      xor,abstract,aliased,protected,requeue,tagged,until},
  sensitive=f,
  morecomment=[l]--,
  morestring=[d]",
  showstringspaces=false,
  basicstyle=\small\ttfamily,
  keywordstyle=\bf\small,
  commentstyle=\itshape,
  stringstyle=\sf,
  extendedchars=true,
  columns=[c]fixed
}

% CI-DESSOUS: conversion des caractères accentués UTF-8 
% en caractères TeX dans les listings...
\lstset{
  literate=%
  {À}{{\`A}}1 {Â}{{\^A}}1 {Ç}{{\c{C}}}1%
  {à}{{\`a}}1 {â}{{\^a}}1 {ç}{{\c{c}}}1%
  {É}{{\'E}}1 {È}{{\`E}}1 {Ê}{{\^E}}1 {Ë}{{\"E}}1% 
  {é}{{\'e}}1 {è}{{\`e}}1 {ê}{{\^e}}1 {ë}{{\"e}}1%
  {Ï}{{\"I}}1 {Î}{{\^I}}1 {Ô}{{\^O}}1%
  {ï}{{\"i}}1 {î}{{\^i}}1 {ô}{{\^o}}1%
  {Ù}{{\`U}}1 {Û}{{\^U}}1 {Ü}{{\"U}}1%
  {ù}{{\`u}}1 {û}{{\^u}}1 {ü}{{\"u}}1%
}

%%%%%%%%%%
% COMMANDES PERSONNALISEES
\newcommand{\shellcmd}[1]{\\\indent\indent\texttt{\footnotesize\# #1}\\}

%%%%%%%%%%
% TAILLE DES PAGES (A4 serré)

\setlength{\parindent}{0pt}
\setlength{\parskip}{1ex}
\setlength{\textwidth}{17cm}
\setlength{\textheight}{24cm}
\setlength{\oddsidemargin}{-.7cm}
\setlength{\evensidemargin}{-.7cm}
\setlength{\topmargin}{-.5in}

%%%%%%%%%%
% EN-TÊTES ET PIED DE PAGES

%% \pagestyle{fancyplain}
\renewcommand{\headrulewidth}{0pt}
\addtolength{\headheight}{1.6pt}
\addtolength{\headheight}{2.6pt}
\lfoot{}
\cfoot{\footnotesize\sf TPL Algo2 Decomposition Polygones}
\rfoot{\footnotesize\sf page~\thepage/\pageref{LastPage}}
\lhead{}


%%%%%%%%%%
% TITRE DU DOCUMENT

\title{Rapport d'Algo2\\
  ``{\em Decomposition en polygones monotones}'' }
\author{\textsc{Guillaume Halb (G5)} - \textsc{Pierre Thalamy (G1)}}
\date{\today}

\begin{document}

\maketitle

\section{Vue d'ensemble du projet}
\subsection{Compilation et execution}
Pour commencer, notre programme respecte bien les consignes et le
cahier des charges décrit dans l'énoncé. De plus, il décompose avec
succés les polygones décrits par les fichier \emph{.in} fournis.

Notre executable prend deux arguments :
\shellcmd{./decomposition\_polynomes data.in output.svg}
Avec : 
\begin{itemize}
  \item data.in : Le fichier d'entrée décrivant le polygone à
    monotoniser.
  \item output.svg : Le nom du fichier svg sur lequel sera dessiné le
    polygone + les segments de monotonisation (aussi affichés sur
    Stdout par respect pour l'énoncé).
\end{itemize}
Après avoir compilé la commande suivante dans le repertorie \emph{Decpoly/} :
\shellcmd{gnatmake decomposition\_Polynomes.adb}

\subsection{Organisation du code}

Le projet comprend 7 modules :
\begin{enumerate}
\item \emph{Decomposition\_Polynomes} : Le main faisant appels aux procédures de tous
  les autres modules.
\item \emph{Defs} : Comprend les définitions de tous les objets et
  structures de données utilisées dans le projet (Arbre exclu), ainsi que diverses
  procédures de manipulation des types.
\item \emph{Generic\_ABR} : Un package générique de gestion d'arbres
  binaires de recherche, tel que décrit dans l'énoncé. 
\item \emph{Liste} : Liste doublement chainée pour le stockage des
  segments.
\item \emph{Parseur} : Procédures d'extraction des données des fichiers
  d'entrée pour stockage dans les structures appropriées.
\item \emph{Traitement} : Contient les procédures d'exploitation des
  données stockées (Génération des segments, tri du
  tableau par ordre lexicographique, algorithme de traitement des point).
\item \emph{Svg} : Réalise le du polygone, puis des segments de monotonisation.
\end{enumerate}

\newpage

\subsection{Exécution séquentielle du programme}
Nous décrivons ci-dessous les actions exécutées par le programme de
façon séquentielle, comme il est possible de le voir très facilement
dans notre fichier \emph{main}.

\begin{enumerate}
\item \emph{Lecture des données d'entrée} : Via le parseur, on
  instancie un élément de type sommet pour chaque ligne du fichier
  d'entrée et on les stocke à la suite dans le tableau de pointeurs sur sommets.
\item \emph{Initialisation des segments entrants / sortants} : Pour
  chaque élément du tableau de sommet, on génére les listes de
  segments entrants et sortants suivant l'ordre lexicographique.
\item \emph{Dessin du polygone} : On écrit le code du polygone dans le
  fichier SVG passé en argument.
\item \emph{Tri du tableau selon l'ordre lexicographique} : On utilise
  le Generic Array Sort de la bibliothèque Ada pour trier le tableau
  de sommet selon l'ordre lexicographique.
\item \emph{Exécution de l'algorithme de décomposition} : Pour chaque
  point du tableau, on applique l'algorithme de décomposition donné
  dans le sujet. Les segments de monotonisation générés sont stockés
  dans une liste.
\item \emph{Export SVG des segments de monotisation} : On ajoute les
  segments générés par l'algorithme au fichier SVG et on le ferme.
\end{enumerate}

\section{Choix de conception}
\subsection{Modélisation et stockage des sommets}
Nous avons défini un sommet comme un type comprenant :
\begin{itemize}
  \item Une lettre representant le sommet, en fonction de son ordre d'apparition
    dans le fichier d'entrée. 
  \item Ses coordonnnées, modélisé par le type Position.
  \item Deux listes doublement chainées, une pour les segments
    entrants, et une autre pour les segments sortants.
\end{itemize}
C'est vrai qu'il est étrange d'avoir choisi une liste pour stocker un
nombre fini connu d'éléments, mais comme nous utilisons aussi une liste
de segments pour stocker les segments de monotonisation, dont le
nombre est inconnu, autant réutiliser cette structure. \\
Par ailleurs, dans la quasi intégralité du code, nous manipulons des
pointeurs sur sommets plutôt que les sommets eux même. Les sommets
sont stockés et accessibles via un tableau de pointeurs sur sommets,
et un segment est défini comme un couple de deux pointeurs sur
sommet. Cela est dû au fait que si nous passions les sommets par copie
aux différentes procédures, les tests d'égalité entre eux seraient
impossible, leurs listes de segments changeant sans arrêt. \\
Enfin, en réalité nous ne manipulons directement le tableau de
sommets, mais un pointeur sur tableau de sommets dynamique,
\emph{TSom\_Ptr}, afin de ne pas avoir à se soucier des contraintes
d'initialisation du tableau dynamique.

\subsection{Arbre Binaire de Recherche}
Pour l'arbre binaire de recherche, nous avons suivi globalement suivi
l'implémentation faite en cours, tout en rajoutant les procédures
requises par l'énoncé (\emph{Compte\_Position()} et
\emph{Noeuds\_Voisins()}). De plus, nous proposons une procédure
d'export au format \emph{.dot} et d'affichage sur stdout de l'arbre.\\
Notre ABR a été implémenté dans un package générique nécessitant une
fonction de comparaison pour le \emph{Type\_Clef}, et une fonction
d'affichage d'un élément de ce type sur stdout (Pour prise en charge
de l'affichage et export dot). Nous avons testé notre implémentation
avec des entiers naturels grâce au fichier \emph{test\_abr.adb}. \\
En ce qui concerne l'implémentation de la fonction \emph{``au dessus
  de''} pour le type segment, nous avons trouvé que la manière la plus
simple de comparer l'ordonné de deux segments était de comparer
l'ordonné de leur milieu.

\end{document}
